\documentclass{ecnreport}

\stud{Master 1 CORO / Option Robotique}
\topic{Robot Operating System}
\author{O. Kermorgant}

\begin{document}

\inserttitle{Robot Operating System}

\insertsubtitle{Lab 2: the ``mirror arm'' node}

\section{Goals}

Program a node that can move the left arm of the Baxter robot in a symmetric way with respect to
the motion of the right arm (symmetry with respect to the sagittal plane of the robot).
The node should be usable in position mode (the joint positions of the master arm are `` copied to ''
the slave arm) or in velocity mode (the joint velocities of the master arm are `` copied to '' the slave
arm).

\section{Information}

To create the control loop you must get the current \emph{state} of the right arm and send \emph{commands} to 
the left arm. To not hesitate to use RViz to compare the frames, some of the joints need to get the negative
value of the other arm.

\subsection{Tasks}

\begin{itemize}
\item Identify the topics that the node should subscribe and publish to.
 \item Create a ROS package (\texttt{catkin create pkg <name> --catkin-deps <dependencies>}) with dependencies on \texttt{baxter\_core\_msgs}, 
 \texttt{sensor\_msgs} and \texttt{ecn\_common}
 \item Draw the expected graph of the application. 
 \item Program the node in C++ and/or Python
 \end{itemize}

Once the package is created, feel free to use the provided \texttt{lab2\_mirror.cpp} file as a template for the node code.
 





\end{document}
