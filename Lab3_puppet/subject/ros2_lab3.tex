\documentclass{ecnreport}

\stud{Master 1 CORO / Option Robotique}
\topic{Robot Operating System}
\author{O. Kermorgant}

\begin{document}

\inserttitle{Robot Operating System}

\insertsubtitle{Lab 3: the ``puppet arm'' node}

\section{Goals}

In this lab, again the left arm will be controlled depending on the position of the right arm.\\
This time, a constant 3D transform will be imposed between the left and right grippers. 

The simulation should be started in a ROS 1 console with:
\begin{bashcodelarge}
 roslaunch baxter_simple_sim simulation.launch lab:=puppet
\end{bashcodelarge}

\section{ROS concepts}

\subsection{Publishing to a topic}

In order to control the left arm, you need to publish a command message on the suitable topic (as in the lab 2)

\subsection{Using TF}

In ROS, the \texttt{/tf} topic conveys many 3D transforms between frames, forming a tree. A TF listener can be instantiated in a node in order to retrieve any transform between two frames. \\
In our case, the desired pose of the left gripper will be defined as a new frame,,  relative to the right gripper and called \texttt{'left\_gripper\_desired'}.\\
The node \texttt{static\_transform\_publisher} from the \texttt{tf2\_ros} package is designed to publish this kind of fixed transform between frames:
\begin{bashcodelarge}
 ros2 run tf2_ros static_transform_publisher x y z yaw pitch roll frame_id child_frame_id
\end{bashcodelarge}where:
\begin{itemize}
\item \texttt{frame\_id} is the reference frame (here \texttt{right\_gripper})
\item \texttt{child\_frame\_id} is the target frame (here \texttt{left\_gripper\_desired})
 \item \texttt{x y z yaw pitch roll} is the 3D transform (here \texttt{0 0 0.1 0 3.14 0})
\end{itemize}


For \texttt{/tf} to get all the transforms for Baxter, a \texttt{robot\_state\_publisher} must be run. It is a standard node that:
\begin{itemize}
 \item loads the description of a robot (URDF file)
 \item subscribes to the joint states topic
 \item publishes all induced 3D transforms with the direct geometric model
\end{itemize}
A custom launch file already exists to do this and can be run with:
\begin{bashcodelarge}
 ros2 launch baxter_description baxter_state_publisher_launch.py
\end{bashcodelarge}


A single launch file should be written in order to regroup the two given commands:
\begin{itemize}
 \item run the node \texttt{static\_transform\_publisher} with the correct arguments
 \item include the launch file \texttt{baxter\_state\_publisher\_launch.py}
 \end{itemize}
 
 With the simulation and your launch file running, check that you can indeed retrieve the current 3D transform between the frames \texttt{base} and \texttt{left\_gripper\_desired}:
 \begin{bashcodelarge}
 ros2 run tf2_ros tf2_echo base left_gripper_desired
\end{bashcodelarge}



\section{Tasks}

\begin{itemize}
\item Identify the topics that the node should publish to: names, message type
\item Identify the services that the node need in order to convert 3D transform to joint positions: names, service type
\item Check the online documentation ``ROS 2 C++ services'' to get the overall syntax. This work requires at least a publisher, a timer, tf, and service client
\item Program the node in C++ (and then in Python3 if you feel like it)
\end{itemize}

The package is already created for this lab (\texttt{lab3\_puppet}), you just have to update the C++ file and compile it.\\
Feel free to keep this package as a template for future packages / nodes that you will create.
\end{document}
